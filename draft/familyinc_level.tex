\begin{table}
  \centering
  \caption{Family income level classification.}
  \label{familyinc_level}
  \begin{tabular}{rl}
  \hline
  Level & Income               \\ \hline
  1     & Less than \$5,000    \\
  2     & \$5,000 to \$7,499   \\
  3     & \$7,500 to  \$9,999  \\
  4     & \$10,000 to \$12,499 \\
  5     & \$12,500 to \$14,999 \\
  6     & \$15,000 to \$19,999 \\
  7     & \$20,000 to \$24,999 \\
  8     & \$25,000 to \$29,999 \\
  9     & \$30,000 to \$34,999 \\
  10    & \$35,000 to \$39,999 \\
  11    & \$40,000 to \$49,999 \\
  12    & \$50,000 to \$59,999 \\
  13    & \$60,000 to \$74,999 \\
  14    & \$75,000 or more     \\ \hline
  \end{tabular}
  \subcaption*{In the raw CPS data, family income refers to combined income of all family members during the last 12 months, including both labour and non-labour incomes. Starting in October 2003, additional levels are introduced to classify the ranges \$75,000 to 99,999 and \$100,000 to 149,999. However, for this paper's analysis, those new levels are included in Level 14 for consistency with older data.}
\end{table}